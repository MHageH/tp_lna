\documentclass[a4paper]{article}
%\usepackage[T1]{fontenc}
\usepackage[english]{babel}

\usepackage{amsmath}
\usepackage{amssymb,amsfonts,textcomp, graphicx}
\usepackage{graphics}

\usepackage{wrapfig}

\usepackage{parskip}

\usepackage{color}
\usepackage{array}
\usepackage{hhline}
\usepackage{subcaption}

\usepackage{textcomp}

\usepackage[hidelinks]{hyperref}

\setlength\tabcolsep{1mm}
\renewcommand\arraystretch{1.3}

\setlength\voffset{-1in}
\setlength\hoffset{-1in}
\setlength\topmargin{0.7874in}
\setlength\oddsidemargin{0.7874in}
\setlength\textheight{10.118099in}
\setlength\textwidth{6.6932993in}
\setlength\footskip{0.0cm}
\setlength\headheight{0cm}
\setlength\headsep{0cm}


\begin{document}

\newcommand\textstyleEmphasis[1]{\textit{#1}}
\renewcommand{\contentsname}{Table des mati\`eres}
\renewcommand\refname{R\'ef\'erences}

\renewcommand{\abstractname}{Pr\'eambule}
\title{\textbf{Projet Circuits Int\'egr\'es Radiofr\'equence \\ Conception d'un LNA \`a 2.45 GHz \\ en Technologie 0.35 $\mu$m AMS}}
\author{Mohamed Hage Hassan \\ Cl\'ement Cheung}
\date{6 D\'ecembre 2017}
\maketitle
\thispagestyle{empty}

\tableofcontents
\clearpage

\iffalse

\begin{figure}[!htb]
\begin{center}
  \includegraphics[scale=0.47]{Echantillonneur-bloqueur.png}
  \caption{Sch\'ema d'un \'echantilloneur-bloqueur \`a capacit\'e commut\'ee}
\end{center}
\end{figure}

\begin{figure}[!htb]
  \begin{subfigure}[t]{.5\linewidth}
      \centering
      \includegraphics[width=1.1\linewidth]{circuit-RC.png}
      \label{fig:rccircuit}
  \end{subfigure}%
  \begin{subfigure}[t]{.5\linewidth}
    \centering
    \includegraphics[width=1.1\linewidth]{sim-inital.png}
    \label{fig:rccircuit-sim}
  \end{subfigure}%
  \caption{Sch\'ema et Simulation du circuit}
  \label{fig:RC-sim}
\end{figure}

\fi

\section*{Introduction}
\addcontentsline{toc}{section}{Introduction}



\section*{Conclusion}
\addcontentsline{toc}{section}{Conclusion}




\addcontentsline{toc}{section}{R\'ef\'erences}
\begin{thebibliography}{9}

\bibitem{RFIC-cours}
\textit{Radio Frequency Integrated Circuits Course}\\
\texttt{Sylvain Bourdel, Florence Podevin, Institut Polytechnique de Grenoble - Phelma}

\bibitem{conception-adaptation}
\textit{Conception d'un circuit en L \`a l'aide de l'abaque de Smith}\\
\texttt{http://f5zv.pagesperso-orange.fr/RADIO/RM/RM23/RM23p/RM23p03.html}

\bibitem{Analog-CMOS-microelectronics}
\textit{Design of Analog CMOS Integrated Circuits, 2nd Edition}\\
\texttt{Behzad Razavi, McGraw-Hill Education}

\bibitem{conception-circuits-integrees}
\textit{Conception de circuits int\'egr\'es analogique}\\
\texttt{Laurent Aubard, Institut Polytechnique de Grenoble - Phelma}

\end{thebibliography}


\end{document}
